\documentclass[12pt, a4paper]{article}
\usepackage[french]{babel}
\usepackage{helvet}
\usepackage[T1]{fontenc}
\usepackage[utf8]{inputenc}
\usepackage{geometry}
\usepackage{fancyhdr}
\usepackage{titlesec}
\usepackage{array}
\usepackage{graphicx}
\usepackage{amsmath}
\usepackage{xcolor}
\usepackage{enumitem}
\usepackage{hyperref}
\usepackage{booktabs}

\geometry{margin=2.5cm}
\titleformat{\section}{\large\bfseries}{\thesection}{1em}{}

\pagestyle{fancy}
\fancyhf{}
\fancyhead[L]{\small\textbf{Résumé des notes des exposés}}
\fancyhead[R]{\small\textbf{Théories et Pratiques de l'Investigation Numérique}}
\fancyfoot[C]{\thepage}


\begin{document}

% Page de garde
\begin{titlepage}
	\begin{tabular}{p{0.42\linewidth} p{0.2\linewidth} p{0.4\linewidth}}
		\centering
		\begin{small}
		\textbf{RÉPUBLIQUE DU CAMEROUN}                   \\
		******                                            \\
		Paix \textendash{} Travail \textendash{} Patrie   \\
		******                                            \\
		\textbf{UNIVERSITÉ DE YAOUNDÉ I}                  \\
		******                                            \\
		ÉCOLE NATIONALE SUPÉRIEURE                        \\
		POLYTECHNIQUE DE YAOUNDÉ                          \\
		******                                            \\
		\textbf{DEPARTEMENT DE GENIE INFORMATIQUE}        \\
		******
		\end{small}
		 &
		\includegraphics[width = 0.8 \linewidth]{Logo/Logo_ENSPY.png}
		 &
		\centering
		\begin{small}
		\textbf{REPUBLIC OF CAMEROON}                     \\
		******                                            \\
		Peace \textendash{} Work \textendash{} Fatherland \\
		******                                            \\
		\textbf{UNIVERSITY OF YAOUNDÉ I}                  \\
		******                                            \\
		NATIONAL ADVANCED SCHOOL                          \\
		ENGINEERING OF YAOUNDE                            \\
		******                                            \\
		\textbf {COMPUTERS ENGINEERING DEPARTMENT}        \\
		******
		\end{small}                                            \\
	\end{tabular}
	\centering
	\vspace*{1cm}

	% Titre principal
	\framebox[\textwidth]{
		\parbox{0.9\textwidth}{
			\centering
			\vspace{0.8cm}
			\Huge\textbf{TRAVAIL À FAIRE N°4}
			\vspace{0.8cm}
		}
	}

	\vspace{1cm}

	% Sous-titre
	{\Large Exercices du chapitre 2}

	\vspace{0.5cm}

	% Titre du cours
	{\LARGE \textbf{Théories et Pratiques de l'Investigation Numérique}}

	\vspace{2cm}

	% Section auteurs
	{\large Rédigé par :}

	\vspace{0.5cm}

	\begin{tabular}{|>{\centering\arraybackslash}m{8cm}
		|>{\centering\arraybackslash}m{4cm}
		|>{\centering\arraybackslash}m{3cm}|}
		\hline
		\textbf{Noms \& Prénoms} & \textbf{Filière} & \textbf{Matricule} \\
		\hline
		DSAMAGO JAFFO Trésor     & HN -- CIN 4      & 22P036             \\
		\hline
	\end{tabular}

	\vfill

	% Encadrement
	\begin{Large}
		\textbf{Sous la direction de :} \\
		M. \textbf{MINKA MI NGUIDJOI Thierry Emmanuel} \\
	\end{Large}

	\vspace{1cm}

	% Année académique
	\textbf{Année académique : 2025--2026}
\end{titlepage}
\section*{Partie 1 : Analyse Historique et Épistémologique}

\subsection*{Exercice 1 : Analyse Comparative des Régimes de Vérité}

\subsubsection*{Choix des périodes : 1990-2000 vs 2010-2020}

\textbf{Vecteurs de dominance :}

\begin{itemize}
\item \textbf{Période 1990-2000} : \(\vec{R} = (0.3, 0.4, 0.1, 0.2)\)
  \begin{itemize}
  \item Dominance juridique (\(\alpha_J = 0.4\)) : Création de cadres légaux (CFAA, etc.)
  \item Technique (\(\alpha_T = 0.3\)) : Internet émergent
  \item Pratique (\(\alpha_P = 0.2\)) : Standardisation des méthodes
  \item Social (\(\alpha_S = 0.1\)) : Élite technique
  \end{itemize}
  
\item \textbf{Période 2010-2020} : \(\vec{R} = (0.4, 0.2, 0.2, 0.2)\)
  \begin{itemize}
  \item Dominance technique (\(\alpha_T = 0.4\)) : Big Data, Cloud, IA
  \item Équilibre relatif des autres dimensions
  \end{itemize}
\end{itemize}

\subsubsection*{Discontinuités épistémologiques (Foucault)}

\begin{itemize}
\item \textbf{1990-2000} : Passage de l'artisanat à la profession
\item \textbf{2010-2020} : Passage de l'humain à l'algorithmique
\end{itemize}

\subsubsection*{Explication sociotechnique}

La transition fut \textbf{revolutionnaire} car elle a impliqué un changement de paradigme épistémique : la vérité n'est plus établie principalement par des experts humains mais par des systèmes algorithmiques.

\subsection*{Exercice 2 : Étude de Cas Archéologique Foucaldienne}

\subsubsection*{Analyse de l'affaire Enron (2001) comme formation discursive}

\textbf{Contexte du « dicible » et « pensable » en 2001 :}

\begin{itemize}
\item \textbf{Dicible} : « preuve électronique », « email comme evidence », « analyse de données structurées »
\item \textbf{Impossible à énoncer} : « intelligence artificielle », « blockchain », « preuve algorithmique auto-validante »
\item \textbf{Pensable} : La vérification humaine comme garante ultime de la vérité
\item \textbf{Impensable} : Que des algorithmes puissent établir des vérités juridiques sans supervision humaine
\end{itemize}

\textbf{Régime de vérité en action :}

\[
P_{algo} = \mathcal{A}(\mathcal{D}) \in \mathcal{P}_{légitime} \quad si\ \mathcal{V}(\mathcal{A}) > \theta
\]

\begin{itemize}
\item \textbf{Autorité épistémique} : Expert technique + Tribunal
\item \textbf{Preuve paradigmatique} : Document électronique avec métadonnées
\item \textbf{Institutions légitimes} : Cours de justice, firmes d'audit
\end{itemize}

\subsubsection*{Comparaison avec une affaire contemporaine : SolarWinds (2020)}

\textbf{Évolution du régime :}
\begin{itemize}
\item \textbf{Preuve} : Analyse comportementale IA vs documents électroniques
\item \textbf{Autorité} : Algorithmes d'IA vs experts humains
\item \textbf{Validation} : Apprentissage automatique vs procédures standards
\end{itemize}

\subsection*{Partie 2 : Modélisation Mathématique et Prospective}

\subsection*{Exercice 3 : Modélisation de l'Évolution des Régimes}

\subsubsection*{Modèle mathématique d'évolution}

Soit l'équation de transition :
\[
\vec{R}_{t+1} = F(\vec{R}_t, \Delta Tech_t, \Delta Legal_t, \mathcal{I}_t)
\]

Avec :
\begin{align*}
F(\vec{R}_t, \Delta Tech_t, \Delta Legal_t, \mathcal{I}_t) &= \vec{R}_t + \alpha \cdot \Delta Tech_t \cdot \vec{T} + \beta \cdot \Delta Legal_t \cdot \vec{J} \\
&\quad + \gamma \cdot \mathcal{I}_t \cdot \vec{S} + \delta \cdot \nabla P_t
\end{align*}

Où :
\begin{itemize}
\item $\vec{T} = (1, 0, 0, 0)$ : vecteur unitaire technologique
\item $\vec{J} = (0, 1, 0, 0)$ : vecteur unitaire juridique
\item $\vec{S} = (0, 0, 1, 0)$ : vecteur unitaire social
\item $\nabla P_t$ : gradient des pratiques professionnelles
\end{itemize}

\subsubsection*{Simulation de transition 2010 → 2020}

Données empiriques :
\begin{align*}
\Delta Tech_{2010} &= 0.8 \quad \text{(Big Data, Cloud, IA)} \\
\Delta Legal_{2010} &= 0.3 \quad \text{(RGPD, lois cybersécurité)} \\
\mathcal{I}_{2010} &= 0.6 \quad \text{(Snowden, Cambridge Analytica)} \\
\nabla P_{2010} &= (0.1, 0.1, 0.2, 0.6)
\end{align*}

Application :
\[
\vec{R}_{2020} = (0.3, 0.4, 0.1, 0.2) + 0.8 \cdot (1,0,0,0) + 0.3 \cdot (0,1,0,0) + 0.6 \cdot (0,0,1,0) + (0.1, 0.1, 0.2, 0.6)
\]

\subsection*{Exercice 4 : Vérification de l'Accélération Technologique}

\subsubsection*{Dates des changements de régime}

\begin{table}[h]
\centering
\begin{tabular}{lll}
\toprule
Transition & Période & Durée (années) \\
\midrule
R1 → R2 & 1990-2000 & 10 \\
R2 → R3 & 2000-2010 & 10 \\
R3 → R4 & 2010-2020 & 10 \\
R4 → R5 & 2020-2025 & 5 \\
\bottomrule
\end{tabular}
\caption{Durées des transitions entre régimes}
\end{table}

\subsubsection*{Vérification de la loi d'accélération}

Équation : $\Delta t_{n+1} = k \cdot \Delta t_n$

Avec les données :
\begin{align*}
\Delta t_2 &= k \cdot \Delta t_1 \Rightarrow 10 = k \cdot 10 \Rightarrow k = 1 \\
\Delta t_3 &= k \cdot \Delta t_2 \Rightarrow 10 = k \cdot 10 \Rightarrow k = 1 \\
\Delta t_4 &= k \cdot \Delta t_3 \Rightarrow 5 = k \cdot 10 \Rightarrow k = 0.5
\end{align*}

\textbf{Conclusion :} L'accélération est significative à partir de 2020 ($k = 0.5$).

\subsection*{Exercice 5 : Analyse du Trilemme CRO Historique}

\subsubsection*{Scores CRO par période}

\begin{table}[h]
\centering
\begin{tabular}{lccc}
\toprule
Période & Confidentialité (C) & Fiabilité (R) & Opposabilité (O) \\
\midrule
1970-1990 & 0.8 & 0.4 & 0.3 \\
1990-2000 & 0.6 & 0.7 & 0.6 \\
2000-2010 & 0.4 & 0.8 & 0.8 \\
2010-2020 & 0.3 & 0.9 & 0.7 \\
2020-... & 0.2 & 0.95 & 0.6 \\
\bottomrule
\end{tabular}
\caption{Évolution historique du trilemme CRO}
\end{table}

\subsubsection*{Équation du trilemme}


Preuve par l'absurde :
\begin{itemize}
\item Confidentialité parfaite ($C=1$) implique impossibilité de vérification externe
\item Contredit l'opposabilité parfaite ($O=1$) qui nécessite transparence
\item Donc $\neg(C=1 \land O=1)$
\end{itemize}

\section*{Partie 3 : Investigation Historique Appliquée}

\subsection*{Exercice 6 : Reconstruction Archéologique d'Investigation}

\subsubsection*{Affaire Kevin Mitnick (1995) - Reconstruction historique}

\textbf{Outils et méthodes de 1995 :}
\begin{itemize}
\item Analyse manuelle des logs système
\item Traçage IP basique
\item Méthodes artisanales de préservation
\item Expertise individuelle prédominante
\end{itemize}

\textbf{Régime de vérité 1995 :}
\begin{itemize}
\item Preuve : Logs systèmes, traces réseau
\item Autorité : Expert technique reconnu
\item Validation : Réputation personnelle
\end{itemize}

\subsubsection*{Réanalyse avec outils modernes}

\textbf{Approche contemporaine :}
\begin{itemize}
\item Analyse automatique des métadonnées
\item Corrélation multi-sources
\item Chaine de custody numérique automatisée
\item Validation cryptographique
\end{itemize}

\textbf{Comparaison des régimes :}
\begin{itemize}
\item 1995 : Vérité par expertise humaine
\item 2024 : Vérité par validation algorithmique
\end{itemize}

\subsection*{Exercice 7 : Projet de Recherche Archéologique}

\subsubsection*{« Trou » identifié : Transition 2008-2012}

\textbf{Hypothèse :} La crise financière de 2008 a accéléré l'adoption des méthodes d'investigation numérique dans le secteur financier.

\textbf{Sources primaires :}
\begin{itemize}
\item RFC sur les standards de sécurité (2008-2012)
\item Publications du NIST post-2008
\item Rapports d'audit financier
\end{itemize}

\textbf{Méthode archéologique :}
\begin{enumerate}
\item Identifier les formations discursives émergentes
\item Cartographier les réseaux d'acteurs
\item Analyser les ruptures épistémiques
\end{enumerate}

\subsection*{Exercice 8 : Analyse Prospective des Régimes Futurs}

\subsubsection*{Scénario 2030 : Régime Neuro-Digital}

\textbf{Conditions de possibilité :}
\begin{itemize}
\item Interfaces cerveau-machine matures
\item Standards de neuro-éthique établis
\item Cadre juridique des preuves neurales
\end{itemize}

\textbf{Méthodologie d'investigation :}
\begin{itemize}
\item Analyse des patterns neuronaux
\item Validation par corrélation neuro-comportementale
\item Standards de préservation des traces neurales
\end{itemize}

\textbf{Défis éthiques :}
\begin{itemize}
\item Confidentialité des pensées
\item Authenticité des souvenirs
\item Consentement éclairé
\end{itemize}
\end{document}