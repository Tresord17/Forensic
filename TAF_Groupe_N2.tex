\documentclass[12pt, a4paper]{article}
\usepackage[french]{babel}
\usepackage{helvet}
\usepackage[T1]{fontenc}
\usepackage[utf8]{inputenc}
\usepackage{geometry}
\usepackage{fancyhdr}
\usepackage{titlesec}
\usepackage{array}
\usepackage{graphicx}
\usepackage{xcolor}
\usepackage{enumitem}
\usepackage{hyperref}
\usepackage{booktabs}

\geometry{margin=2.5cm}
\setlength{\headheight}{20pt}

\pagestyle{fancy}
\fancyhf{}
\fancyhead[L]{\small\textbf{}}
\fancyhead[R]{\small\textbf{Théories et Pratiques de l'Investigation Numérique}}
\fancyfoot[C]{\thepage}

\titleformat{\section}{\large\bfseries}{\thesection}{1em}{}
\titleformat{\subsection}{\bfseries}{\thesubsection}{1em}{}

\begin{document}

% Page de garde
\begin{titlepage}
	\begin{tabular}{p{0.42\linewidth} p{0.2\linewidth} p{0.4\linewidth}}
		\centering
		\begin{small}
		\textbf{RÉPUBLIQUE DU CAMEROUN}                   \\
		******                                            \\
		Paix \textendash{} Travail \textendash{} Patrie   \\
		******                                            \\
		\textbf{UNIVERSITÉ DE YAOUNDÉ I}                  \\
		******                                            \\
		ÉCOLE NATIONALE SUPÉRIEURE                        \\
		POLYTECHNIQUE DE YAOUNDÉ                          \\
		******                                            \\
		\textbf{DEPARTEMENT DE GENIE INFORMATIQUE}        \\
		******
		\end{small}
		 &
		\includegraphics[width = 0.8 \linewidth]{Logo/Logo_ENSPY.png}
		 &
		\centering
		\begin{small}
		\textbf{REPUBLIC OF CAMEROON}                     \\
		******                                            \\
		Peace \textendash{} Work \textendash{} Fatherland \\
		******                                            \\
		\textbf{UNIVERSITY OF YAOUNDÉ I}                  \\
		******                                            \\
		NATIONAL ADVANCED SCHOOL                          \\
		ENGINEERING OF YAOUNDE                            \\
		******                                            \\
		\textbf {COMPUTERS ENGINEERING DEPARTMENT}        \\
		******   
		\end{small}                                         \\
	\end{tabular}
	\centering
	\vspace*{1cm}

	% Titre principal
	\framebox[\textwidth]{
		\parbox{0.9\textwidth}{
			\centering
			\vspace{0.8cm}
			\Huge\textbf{TRAVAIL À FAIRE}
			\vspace{0.8cm}
		}
	}

	\vspace{1cm}

	% Sous-titre
	{\LARGE \textbf{Simulation d'une serie de messages sur WhatsApp entre un homme et sa maitresse}}

	\vspace{0.5cm}

	% Titre du cours
	{\LARGE {Théories et Pratiques de l'Investigation Numérique}}

	\vspace{2cm}

	% Section auteurs
	{\large Rédigé par :}

	\vspace{0.5cm}

	\begin{tabular}{|>{\centering\arraybackslash}m{8cm}
		|>{\centering\arraybackslash}m{4cm}
		|>{\centering\arraybackslash}m{3cm}|}
		\hline
		\textbf{Noms \& Prénoms}   & \textbf{Filière} & \textbf{Matricule} \\
		\hline
		DSAMAGO JAFFO Trésor       & HN -- CIN 4      & 22P036             \\
		\hline
		EMBOLO MVOGO Shawn Douglas & HN -- CIN 4      & 22P072             \\
		\hline
		MELONE Andre               & HN -- CIN 4      & 22P059             \\
		\hline
	\end{tabular}

	\vfill

	% Encadrement
	\begin{Large}
		\textbf{Sous la direction de :} \\
		M. \textbf{MINKA MI NGUIDJOI Thierry Emmanuel} \\
	\end{Large}

	\vspace{0.5cm}

	\vspace{1cm}

	% Année académique
	\textbf{Année académique : 2025--2026}
\end{titlepage}

\tableofcontents
\newpage

\addcontentsline{toc}{section}{Introduction}
\section*{Introduction}
Dans le contexte actuel où les échanges numériques occupent une place centrale dans la vie sociale et personnelle, les applications de messagerie instantanée comme WhatsApp constituent une source d’information privilégiée, mais aussi un vecteur de manipulations. L’investigation numérique vise précisément à analyser, comprendre et parfois reproduire ces environnements afin d’étudier les traces laissées par les utilisateurs ou de mettre en évidence la possibilité de falsifications.
Dans le cadre de ce travail, nous avons choisi de simuler une série de messages échangés entre un homme et sa maîtresse, en mobilisant deux outils : Chatsmock, qui permet de générer de fausses conversations WhatsApp, et Adobe Photoshop, utilisé pour affiner et personnaliser l’apparence des échanges. L’objectif n’est pas de porter un jugement moral sur les faits simulés, mais d’illustrer les possibilités techniques offertes par ces logiciels et de questionner la fiabilité des preuves numériques dans un contexte d’enquête.
\newpage
\section{Mise en situation}
Dans le cadre de cet exercice d’investigation numérique, il a été imaginé le scénario suivant : un enseignant entretiendrait une relation extra-conjugale avec une étudiante de son établissement. Afin de mieux comprendre les enjeux liés à la manipulation des preuves numériques, une simulation de conversations a été réalisée sur l’application WhatsApp. \\

L'enseignant se nomme Paul KENGNE et sa femme Judith KENGNE.

\subsection{Éléments remis pour l'analyse}
Mme KENGNE a fourni les éléments suivants:
\begin{itemize}
	\item Sept (07) captures d’écran extraites de l’application WhatsApp, de discussion récente.
	\item Deux photos de l’élève en tenue d’Adam envoyées via WhatsApp.
\end{itemize}

\subsection{Contenu des échanges}
Les messages révèlent :
\begin{itemize}
	\item Des propos à caractère affectif et sexuel explicite de la part de M. Paul KENGNE.
	\item Des invitations à se retrouver en dehors du cadre scolaire.
	\item Des expressions telles que « Bonsoir mon cœur », « ma femme est une folle », « Je t’aime mon sucre », « mon corps te réclame encore plus ».
	\item L’élève répond avec des messages affectifs.
	\item L’époux promet de quitter sa femme sous peu.
\end{itemize}

\section{Méthodologie de falsification des échanges}
Pour réaliser la simulation, deux outils principaux ont été utilisés : \textbf{Chatsmock} et \textbf{Adobe Photoshop}.  

\subsection{Utilisation de Chatsmock}
Chatsmock est une application web permettant de créer de fausses conversations WhatsApp de manière réaliste.  
Grâce à son interface intuitive, il est possible de :  
\begin{itemize}
	\item Définir les participants à la discussion (noms, photos de profil, numéros de téléphone).  
	\item Générer des messages avec un contenu librement choisi.  
	\item Paramétrer l’heure, la date et le statut de lecture (message envoyé, reçu, ou vu).  
	\item Créer des captures d’écran identiques à celles produites par l’application WhatsApp réelle.  
\end{itemize}
Ainsi, une première version de la conversation fictive a pu être générée, comprenant des échanges textuels imitant un dialogue authentique.  

\subsection{Utilisation de Adobe Photoshop}
Afin d’améliorer le réalisme des captures produites, \textbf{Adobe Photoshop} a ensuite été utilisé pour :  
\begin{itemize}
	\item Corriger certains détails graphiques (alignement, bulles de texte, couleur des icônes).  
	\item Modifier ou insérer des éléments supplémentaires, comme des images envoyées dans la discussion.  
	\item Retoucher l’interface visuelle afin qu’elle corresponde parfaitement à celle d’un smartphone réel.  
\end{itemize}
Ce travail de retouche graphique permet de lever tout soupçon de falsification au premier regard et d’obtenir un rendu quasi identique à une capture d’écran authentique.  

L’association de \textbf{Chatsmock} et de \textbf{Photoshop} démontre qu’il est techniquement simple de fabriquer de fausses preuves numériques. Cela met en évidence les limites de la fiabilité des captures d’écran de messagerie instantanée lorsqu’elles sont présentées comme preuves dans un contexte judiciaire ou disciplinaire.  
\subsection{Pices jointes du rapport}
Ci-jointes, sont les captures d'écran des différentes conversations\\
\includegraphics[width=0.5 \linewidth]{Images TP1/1.jpg}
\includegraphics[width=0.5 \linewidth]{Images TP1/2.jpg}\\
\includegraphics[width=0.5 \linewidth]{Images TP1/3.jpg}
\includegraphics[width=0.5 \linewidth]{Images TP1/4.jpg}\\
\includegraphics[width=0.5 \linewidth]{Images TP1/5.jpg}
\section{Limites de Chatsmock et comparaison avec d’autres outils}

\subsection{Limites de Chatsmock}
Bien que \textbf{Chatsmock} soit un outil simple et efficace pour simuler des conversations WhatsApp, il présente néanmoins certaines limites :  
\begin{itemize}
	\item \textbf{Manque de réalisme sur certains détails} : l’application ne reproduit pas toujours parfaitement l’interface officielle de WhatsApp, notamment en ce qui concerne les mises à jour récentes de design.  
	\item \textbf{Fonctionnalités restreintes} : la personnalisation est limitée à certains paramètres (texte, heure, icônes), mais il est parfois difficile de simuler des éléments complexes comme les notes vocales, les appels ou les réactions aux messages.  
	\item \textbf{Dépendance au format image} : les conversations générées sont souvent uniquement exportées sous forme de captures d’écran, ce qui peut limiter leur usage dans certaines manipulations.  
	\item \textbf{Détection possible par un expert} : malgré le réalisme visuel, une analyse forensique attentive peut identifier des incohérences dans les métadonnées ou des anomalies graphiques.  
\end{itemize}

\subsection{Comparaison avec d’autres outils}
D’autres logiciels et applications permettent également de falsifier des conversations numériques. Comparativement :  
\begin{itemize}
	\item \textbf{FakeChat} : application mobile qui offre plus d’options visuelles, notamment la possibilité de simuler des appels ou des notifications, mais qui reste moins crédible pour un examen expert.  
	\item \textbf{WhatsFake} : similaire à Chatsmock, mais orienté vers une utilisation ludique (blagues, divertissement), avec une interface moins personnalisable.  
	\item \textbf{Photoshop et éditeurs graphiques avancés} : offrent une liberté totale de falsification mais exigent des compétences techniques plus poussées. Ils permettent de modifier n’importe quel détail visuel pour atteindre un degré de réalisme presque indétectable.  
	\item \textbf{Outils forensiques détournés} : certains logiciels conçus pour analyser les bases de données de messagerie peuvent être utilisés à mauvais escient pour injecter ou manipuler directement les enregistrements des conversations.  
\end{itemize}

\subsection{Conclusion partielle}
Ainsi, bien que Chatsmock se distingue par sa facilité d’utilisation et son accessibilité, ses limites en termes de réalisme et de fonctionnalités peuvent être compensées par d’autres outils plus puissants comme Photoshop. La combinaison de ces solutions rend la falsification de conversations de plus en plus sophistiquée, posant un défi croissant aux enquêteurs numériques.  


\section{L'impact de cette categorie d'outils sur l'investigation numérique et quelques recommandations}

\subsection{Impact sur l'investigation numérique}
L’existence et l’accessibilité d’outils tels que \textbf{Chatsmock} et \textbf{Adobe Photoshop} posent des défis majeurs dans le domaine de l’investigation numérique. En effet, ils permettent de créer ou de modifier des preuves numériques en quelques minutes seulement, ce qui entraîne plusieurs conséquences :  
\begin{itemize}
	\item \textbf{Baisse de la fiabilité des captures d’écran} : les images issues d’applications de messagerie ne peuvent plus être considérées comme preuves irréfutables sans vérification complémentaire.  
	\item \textbf{Difficultés pour les experts} : les enquêteurs doivent mobiliser des compétences techniques avancées pour distinguer une preuve authentique d’une preuve falsifiée.  
	\item \textbf{Risque de manipulation judiciaire ou disciplinaire} : des individus mal intentionnés peuvent utiliser ces outils pour nuire à la réputation d’autrui ou influencer une décision juridique.  
	\item \textbf{Multiplication des faux dossiers} : l’usage abusif de tels outils complique la tâche des institutions en augmentant le nombre de preuves potentiellement corrompues.  
\end{itemize}

\subsection{Recommandations}
Face à ces risques, certaines mesures préventives et correctives peuvent être envisagées :  
\begin{itemize}
	\item \textbf{Vérification technique des preuves} : analyser les métadonnées des fichiers (horodatage, signature numérique, origine) afin de confirmer leur authenticité.  
	\item \textbf{Sensibilisation des acteurs judiciaires et administratifs} : former juges, avocats et enquêteurs à la reconnaissance des falsifications numériques.  
	\item \textbf{Utilisation d’outils spécialisés} : recourir à des logiciels de détection de manipulations d’images et d’analyses forensiques.  
	\item \textbf{Préférence pour les données brutes} : privilégier la récupération directe des messages depuis les bases de données des téléphones ou des serveurs, plutôt que de simples captures d’écran.  
	\item \textbf{Renforcement du cadre légal} : établir des règles précises sur l’acceptabilité des preuves numériques devant les juridictions.  
\end{itemize}

L’analyse de ces outils met en lumière la nécessité d’adapter en permanence les méthodes d’investigation numérique. Si la falsification de preuves devient plus facile, les experts doivent redoubler de vigilance et développer des techniques avancées pour garantir la fiabilité des conclusions.  
\newpage
\addcontentsline{toc}{section}{Conclusion}
\section*{Conclusion}
L’expérience de simulation d’une discussion falsifiée sur WhatsApp, réalisée à l’aide des outils \textbf{Chatsmock} et \textbf{Adobe Photoshop}, a permis de démontrer à quel point il est simple de créer des preuves numériques trompeuses. Cette pratique met en évidence la fragilité des éléments de preuve issus des applications de messagerie instantanée, particulièrement lorsque ceux-ci se limitent à de simples captures d’écran. Ce travail illustre ainsi un double constat : d’une part, les menaces que représentent ces falsifications pour la crédibilité des investigations numériques ; d’autre part, la nécessité pour les experts et les instances judiciaires d’adopter des méthodes de vérification rigoureuses. En définitive, l’investigation numérique ne peut se limiter à l’analyse apparente des données. Elle doit intégrer des techniques avancées de vérification, ainsi qu’une sensibilisation accrue des différents acteurs, afin de garantir l’intégrité et la fiabilité des preuves numériques dans un monde où la manipulation devient de plus en plus accessible.  
\newpage

\addcontentsline{toc}{section}{Références}
\subsection*{Références}
\begin{itemize}
    \item \url{https://play.google.com/store/apps/details?id=com.applylabs.whatsmock.free&pcampaignid=web_share}
    \item Théories et Pratiques de l’Investigation Numérique, Mr MINKA Thierry
\end{itemize}
\end{document}