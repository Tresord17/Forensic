\documentclass[12pt, a4paper]{article}
\usepackage[french]{babel}
\usepackage{helvet}
\usepackage[T1]{fontenc}
\usepackage[utf8]{inputenc}
\usepackage{geometry}
\usepackage{fancyhdr}
\usepackage{titlesec}
\usepackage{array}
\usepackage{graphicx}
\usepackage{xcolor}
\usepackage{enumitem}
\usepackage{hyperref}
\usepackage{booktabs}

\geometry{margin=2.5cm}
\setlength{\headheight}{20pt}

\pagestyle{fancy}
\fancyhf{}
\fancyhead[L]{\small\textbf{}}
\fancyhead[R]{\small\textbf{Théories et Pratiques de l'Investigation Numérique}}
\fancyfoot[C]{\thepage}

\titleformat{\section}{\large\bfseries}{\thesection}{1em}{}
\titleformat{\subsection}{\bfseries}{\thesubsection}{1em}{}

\begin{document}

% Page de garde
\begin{titlepage}
	\begin{tabular}{p{0.42\linewidth} p{0.2\linewidth} p{0.4\linewidth}}
		\centering
		\begin{small}
		\textbf{RÉPUBLIQUE DU CAMEROUN}                   \\
		******                                            \\
		Paix \textendash{} Travail \textendash{} Patrie   \\
		******                                            \\
		\textbf{UNIVERSITÉ DE YAOUNDÉ I}                  \\
		******                                            \\
		ÉCOLE NATIONALE SUPÉRIEURE                        \\
		POLYTECHNIQUE DE YAOUNDÉ                          \\
		******                                            \\
		\textbf{DEPARTEMENT DE GENIE INFORMATIQUE}        \\
		******
		\end{small}
		 &
		\includegraphics[width = 0.8 \linewidth]{Logo/Logo_ENSPY.png}
		 &
		\centering
		\begin{small}
		\textbf{REPUBLIC OF CAMEROON}                     \\
		******                                            \\
		Peace \textendash{} Work \textendash{} Fatherland \\
		******                                            \\
		\textbf{UNIVERSITY OF YAOUNDÉ I}                  \\
		******                                            \\
		NATIONAL ADVANCED SCHOOL                          \\
		ENGINEERING OF YAOUNDE                            \\
		******                                            \\
		\textbf {COMPUTERS ENGINEERING DEPARTMENT}        \\
		******   
		\end{small}                                         \\
	\end{tabular}
	\centering
	\vspace*{1cm}

	% Titre principal
	\framebox[\textwidth]{
		\parbox{0.9\textwidth}{
			\centering
			\vspace{0.8cm}
			\Huge\textbf{TRAVAIL À FAIRE}
			\vspace{0.8cm}
		}
	}

	\vspace{1cm}

	% Sous-titre
	{\LARGE \textbf{Simulation d'une serie de messages sur WhatsApp entre un homme et sa maitresse}}

	\vspace{0.5cm}

	% Titre du cours
	{\LARGE {Théories et Pratiques de l'Investigation Numérique}}

	\vspace{2cm}

	% Section auteurs
	{\large Rédigé par :}

	\vspace{0.5cm}

	\begin{tabular}{|>{\centering\arraybackslash}m{8cm}
		|>{\centering\arraybackslash}m{4cm}
		|>{\centering\arraybackslash}m{3cm}|}
		\hline
		\textbf{Noms \& Prénoms}   & \textbf{Filière} & \textbf{Matricule} \\
		\hline
		DSAMAGO JAFFO Trésor       & HN -- CIN 4      & 22P036             \\
		\hline
		EMBOLO MVOGO Shawn DOuglas & HN -- CIN 4      & 22P072             \\
		\hline
		MELONE Andre      & HN -- CIN 4      & 22P059             \\
		\hline
	\end{tabular}

	\vfill

	% Encadrement
	\begin{Large}
		\textbf{Sous la direction de :} \\
		M. \textbf{MINKA MI NGUIDJOI Thierry Emmanuel} \\
	\end{Large}

	\vspace{0.5cm}

	\vspace{1cm}

	% Année académique
	\textbf{Année académique : 2025--2026}
\end{titlepage}

\section{Mise en situation}
Le présent rapport est établi à la demande de \textbf{Judith KENGNE}, elle
aussi enseignante dans la même université, épouse légitime de M. KENGNE, dans
le cadre d’une procédure de divorce pour faute. \textbf{Mme KENGNE} a sollicité
une expertise numérique suite à la découverte de messages compromettants entre
son époux et une étudiante mineure de son établissement.
\section{Elements remis pour l'analyse}
Mme KENGNE a fourni les éléments suivants:
\begin{itemize}
	\item Sept (07) captures d’écran extraites de l’application WhatsApp, de discussion
	      récente.
	\item Deux photos de l’élève en tenue d’Adam envoyées via WhatsApp.
\end{itemize}
\section{Résultats de l’analyse}
\subsection{Authenticité des preuves}
\begin{itemize}
	\item Les captures d’écran sont authentiques, non modifiées, et proviennent du
	      téléphone de Mr Paul.
	\item Le numéro de téléphone de l’étudiante correspond à celui enregistré dans les
	      bases de données de l’institut.
	\item Les messages ont été échangés en dehors des heures de cours, souvent tard dans
	      la nuit
\end{itemize}
\subsection{Contenu des échanges}
Les messages révèlent :
\begin{itemize}
	\item Des propos à caractère affectif et sexuel explicite de la part de M. Paul
	      KENGNE.
	\item Des invitations à se retrouver en dehors du cadre scolaire.
	\item Des expressions telles que « Bonsoir mon cœur », « ma femme est une folle », «
	      Je t’aime mon sucre », « mon corps te réclame encore plus »,
	\item L’élève répond avec des messages affectifs,
	\item L’époux promet de quitter sa femme sous peu
\end{itemize}
\section{Implications juridiques}
Les éléments recueillis peuvent être interprétés comme :
\begin{itemize}
	\item Une violation du devoir conjugal par M. KENGNE.
	\item Une relation inappropriée entre un enseignant et une étudiante, enfreignant la
	      déontologie et susceptible d’engager sa responsabilité disciplinaire et pénale.
	\item Une atteinte à la dignité et potentiellement a la sécurité de Mme KENGNE,
	      justifiant une demande de divorce pour faute.
\end{itemize}
\section{Pices jointes du rapport}
Ci-jointes, sont les captures d'écran des différentes conversations\\
\includegraphics[width=0.5 \linewidth]{Images TP1/1.jpg}
\includegraphics[width=0.5 \linewidth]{Images TP1/2.jpg}\\
\includegraphics[width=0.5 \linewidth]{Images TP1/3.jpg}
\includegraphics[width=0.5 \linewidth]{Images TP1/4.jpg}\\
\includegraphics[width=0.5 \linewidth]{Images TP1/5.jpg}
\includegraphics[width=0.5 \linewidth]{Images TP1/6.jpg}

\end{document}